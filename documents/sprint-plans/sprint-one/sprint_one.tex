\documentclass{article}
\usepackage[utf8]{inputenc}
\usepackage{amssymb}

\title{GAM130 - Team 2 - Sprint 1 Plan}
\author{Christopher Robertson}
\date{31/01/20 - 14/02/20}

\begin{document}

\maketitle

\newpage
\section{Brief Overview}

At the end of this Sprint we want to achieve the main goal of having a base prototype which will include such aspects like the movement and unit spawning which will allow us to have some base functionality to start testing to see how well the game plays. This will also allow us to start the combat aspect of the game in the following sprints too since movement is needed for the combat and is currently a blocker. 
\\
\\
In terms of the modelling the expectation is to have a few models which are created and textured in the low poly format. Furthermore, we are also hoping to have a blockout character done by the end of this sprint so that our animator is fully able to utilise their role.
\\
\\
The concept art is also expected to start producing more concepts for the environment and characters and by the end of the sprint we should have a fair amount of concepts that the modellers should be able to use in the following/future sprints to allow us to start producing more models.
\\
\\
Design have planned to start greyboxing a variety of level backgrounds by the end of this sprint and the amount is expected to be about four. This will assist the modellers in creating their assets and placing them in the scene which should allow us to see our game slowly coming together which is our hope by the end of this sprint. A minor revision of the GDD in terms of programming is also needed just to clarify and double check some aspects to ensure that the programmers and designers are on the same page. The designers also intend to prioritise planning based on player experience which includes acting as if they are the player and planning what needs to be created depending on how relevant and how used it is. We are hoping this will ensure that everything that is important will be created first. A tutorial is also being planned by the designers at the moment too.
\\
\\
In terms of writing, the expectation by the next sprint is to have the character descriptions and Catopedia entries started, but not intended to be finished, or implemented into the engine as of this sprint as the tool for it is still currently in development.
\\
\\
For animation we expect some of the story boards research and development to be completed and the ideal number currently is four story boards completed.
\\
\\
Although we do not have an audio specialist the designers have taken on the role for it and for this sprint will be creating a list of potential sounds required in the game.

\newpage

\section{Tasks}
\subsection{Programming}

\begin{itemize}
\item Pathfinding with Movement done
\item End Turn with Movement Integration
\item Catopedia Done
\item Spawn Menu
\item Spawning Units
\end{itemize}

\subsection{Modelling}

\begin{itemize}
    \item Cat Blockout
    \item Scratching Post
    \item Cardboard Box Fort
    \item Chairs
    \item Weapons
\end{itemize}

\subsection{Art}

\begin{itemize}
    \item Environment Concept Art
    \item One Character Concept Art
\end{itemize}

\subsection{Design}
\begin{itemize}
    \item Four Grey Box Levels
    \item Revising Programming Part of the GDD
    \item Prioritisation by Planning Player Experience
    \item Particle Effects
    \item Tutorial Planning
    \item Paper Prototype Testing
\end{itemize}

\subsection{Writing}

\begin{itemize}
    \item Character Descriptions
    \item Entries for Catopedia
\end{itemize}

\subsection{Animation}

\begin{itemize}
    \item Story Board Research
    \item Four Story Boards Developed
\end{itemize}

\subsection{Audio}
\begin{itemize}
    \item Sound List Creation
\end{itemize}

\end{document}
